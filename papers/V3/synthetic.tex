\section{Dataset} \label{syn_gen_eval}
\subsection{Synthetic Dataset Generation}\label{syn_gen}
In this section, we propose a methodology to generate benchmark multilayer networks with ground truth communities.
% The ground truth communities can be of two types (a) cross layer communities (containing multiple types of
% nodes) (b) single layer communities (containing single type of nodes);
The parameter $\alpha$ regulates the
proportion of cross layer vs single layer communities in the benchmark. The network contains $M$ number of different layers where each
layer $L_i$ contains $N_i$ nodes ($N_i = \left\vert V_i \right\vert$) with average degree $\langle k_i\rangle$.
%The density of coupling links between the layers $L_i$ and $L_j$ is $d_{ij}$.
The methodology contains the following three steps:

\textbf{Step 1. Single layer communities:} First, we apply the LFR benchmark algorithm~\cite{lancichinetti2008benchmark}
to generate communities at
each layer $L_i$ with $N_i$ nodes where both degree and community size distributions follow power law distribution
with exponents $\gamma_i$ and $\beta_i$ respectively. We fix the mixing
parameter as $\mu_i$ to construct $\mathcal{C}_i$ single layer communities in layer $L_i$.

\textbf{Step 2. Cross layer communities:} We combine the community $x_i\in \mathcal{C}_i$ of layer $L_i$ with
community $x_j\in \mathcal{C}_j$ of layer $L_j$ to create the cross layer community $x_{ij}$. Assuming
$\left\vert \mathcal{C}_i \right\vert$ and $\left\vert \mathcal{C}_j \right\vert$ as the number of communities in
layers $L_i$ and $L_j$
respectively, $\left\vert\mathcal{C}_c\right\vert=min\{\left\vert\mathcal{C}_i\right\vert, \left\vert\mathcal{C}_j\right\vert\}$ denotes
the maximum possible number
of cross layer communities. We construct $(|\mathcal{C}_c| \times \alpha)$ cross layer communities by
randomly combining single-layer communities from both the layers $L_i$ \& $L_j$ respectively; notably each
cross layer community $x_{ij}$ may contain one or multiple single layer communities.

\textbf{Step 3. Coupling links:} Finally, we create the coupling links between the layers $L_i$ and $L_j$ with
density $d_{ij}$. Fraction $p$ denotes the mixing parameter for the cross layer communities. We first
distribute  $(N_i \times N_j \times d_{ij})$ coupling links randomly between the layers $L_i$ and $L_j$ where
each link has one end in $L_i$ and another end in $L_j$. Next, we rewire the coupling links such
that $p$ fraction of links stay inside the cross layer communities and the remaining $1-p$ fraction of
links connect different cross layer communities.

%[BM: Need to specify the typical parameters.]

\iffalse

suppose we have $|A|$ nodes in the top layer and $|B|$ nodes in the bottom layer,


For creating coupling links between every two layers,
 we utilise two parameters $d$ and $p$ where $d$ denotes the density of coupling links and $p$ denotes the fraction of such links
 within the mixed
 communities.

///////////////////////////////////////////
 \item Synthetic Dataset: Unlike single layer networks~\cite{lancichinetti2008benchmark}, there is no standard procedure to
 generate synthetic multilayer networks with
 ground truth communities. In ~\cite{Arenas}, the authors proposed such a method for multiplex networks. But it is non-trivial to extend
 it for generic multilayer networks. Hence, we propose a novel technique to generate multilayer networks with ground truth
 communities as described below.

 \begin{enumerate}

  \item \textit{Generating individual layers:} Suppose, we want to generate a network with $L$ layers. According to LFR
  benchmark~\cite{lancichinetti2008benchmark}, we
  generate $L$ subnetworks with ground truth communities. For generating each layer, we take standard LFR parameters such as
  number of nodes, average degree, maximum degree and mixing parameter ($\mu$), as input. The output network has $L$ isolated layers with
  each layer $A$ containing $n_A$ modules.

  \item \textit{Creating multilayer communities:} Once we get individual layers with ground truth communities, we group those
  communities across layers to create multilayer communities (containing nodes of different types).
  For example, suppose we have a 2-layer network where the top layer $A$
  has $n_A$ communities and bottom layer $B$ has $n_B$ communities. The maximum number of multilayer or mixed communities we can create
  is $n_c = min\{n_A , n_B\}$. We take another parameter $\alpha$ to decide the fraction of $n_c$ possible multilayer communities
  we actually create. Finally, we select $(n_C \times \alpha)$ communities from both $A$ and $B$ layers and randomly group
  them pair-wise so that each mixed community contains one single-layer community from both subnetworks $A$ \& $B$.

 \item \textit{Creating coupling links:} For creating coupling links between every two layers,
 we utilise two parameters $d$ and $p$ where $d$ denotes the density of coupling links and $p$ denotes the fraction of such links
 within the mixed
 communities. For the above example, suppose we have $|A|$ nodes in the top layer and $|B|$ nodes in the bottom layer,
 then we first assign
 $(|A| \times |B| \times d)$ crosslayer links randomly where each link has one end in $A$ and another end in $B$.
 After that, we rewire the links such that $p$ fraction of them stay inside the already created
 multilayer communities and the remaining
 $1-p$ fraction of them create cross multilayer-community links.

\end{enumerate}

 The parameter values chosen for generating networks for our experiments are mentioned in Sec.~\ref{eval}.
%  For our experiments, we use 2-layer synthetic networks where each layer is generated with identical LFR parameters. For each layer,
%  the total number of nodes is $100$, the average degree is $6$ and the maximum degree is $10$.
%  %and $\mu \in [0.05,0.40,0.75]$ (same across all layers for a particular network).
%  The other parameters i.e. $\mu$, $\alpha$, $d$ \& $p$ are varied from $0$ to $1$.
%

%  \textcolor{red}{[SP: Can we somehow generate crosslayer communities also with LFR? --- Raphael may check this part.
%  We should also check the multiplex LFR baseline available in website whether we can use it (possibly this is generated using the
%  method introduced in ~\cite{Arenas} ]).
%  }

 \item Real Dataset: Real multilayer datasets with ground truth communities are rarely available.
 So, we show the utility of our algorithm on real networks with the help of a recommendation task.
 First, we create a 2-layer network from an existing Yelp (a Location Based Social Network) academic dataset
 which is made available as part of the Yelp dataset challenge
 \footnote{\url{https://www.yelp.com/dataset\_challenge}}. Utilising that network, we obtain $K$ top recommended locations
 for each user with the help of collaborative filtering[REF]. Finally, in Sec.~\ref{eval}, we evaluate to what extent
 our detected communities can alleviate in such a recommendation task.

 The details of the dataset and the method to create the 2-layer network along with recommended locations, are described below.

 \begin{enumerate}
  \item \textit{Dataset Description:} The available Yelp dataset contains detailed information regarding visitors (Yelp users),
  business units (locations) and the tips (short business descriptions) \&
  reviews (detailed business description) posted by the visitors. Each business unit is indexed by
  a \textit{business ID}, and contains information about its category (restaurants, hotels, shopping etc.),
  latitude-longitude, address etc. as attributes.
  Similarly, each visitor is indexed by \textit{a user ID} and contains information regarding her city, social friendship connections etc.
  The dataset contains information of total $61184$ business units (comprising about $700$ different categories)
  located in $378$ cities and $366715$ visitors (with an average of $7$ social links) during the period $2005-2014$.
%   Each tip \& review entry is properly (daywise) timestamped and indexed by an unique \textit{tip/review id} containing the
%   corresponding visitor and business ID information.
  Additionally, there are nearly 1.6 million reviews and 0.5 million tips in the dataset.

  \item \textit{Creating 2-layer network:} We choose a particular city ``Las Vegas'' from the collection which contains
  $13601$ businesses and $173697$ Visitors visiting those locations. To reduce the size of the network, we first detect
  the maximally connected component in the friendship network of those $173697$ Visitors. This yields a set
  of $244$ connected users. Furthermore, we consider only the $1627$ locations which are at least reviewed once by anyone of them.
%   only the businesses
%   within 5 KM radius from the center of the city. This reduces the number of businesses to $1069$. Furthermore, we consider
%   only the $34563$ visitors who have written at least one review for one of those selected locations.
  After shortlisting the set of users \& locations, we create the layers of the
  network - the bottom layer consists of
  visitors who are connected with each other via friendship links and the top layer contains businesses
  where two businesses are connected
  if and only if they are within ZZ meters of each other.
  Finally, we add crosslayer (coupling) links between the nodes of this two layers with the help of
  reviews i.e. we connect a visitor with a business if she has written at least one review for that business (see Fig.~\ref{yelp}).

  \item \textit{Getting recommendations:} Collaborative filtering is a standard machine learning recommendation
  technique[REF]. Given the collection of visitor-business links,
  collaborative filtering (using business-business similarity) provides us with a set of $K$ top
  recommended (for future visit) businesses for each visitor.
%   Now, the visitors with more than a WW\% pair-wise similarity in their recommended set of
%   businesses are assumed to be liking similar locations and put in the same community. Thus, we create communities comprising of
%   similar minded visitors and the businesses they visited or prefer to visit.
%  [DETAILS]
 \end{enumerate}


\end{enumerate}
\fi


\subsection{Synthetic Dataset Evaluation}\label{syn_evalu}
In the following, we evaluate the performance of the synthetic network generation algorithm; we examine
%In order to verify that the generated networks are constructed properly i.e.
whether the generated networks behave consistently with our expectation. In order to accomplish the task,
we adhere to the following approach -
\textbf{(a)} we generate synthetic networks varying different model parameters $\mu$, $\alpha$ \& $p$, which intrinsically regulate the
quality of the community structures present in the generated networks.
\textbf{(b)} we apply state-of-the-art multilayer community detection algorithms~\cite{metafac, CompMod} on this synthetic networks
and evaluate the quality of the detected communities with respect to the ground truth communities.
\textbf{(c)} we conclude that the synthetic network possesses desired properties, only if the quality of the detected communities
in step (b) is consistent with the quality of ground truth communities specified in step (a).
% By doing so, we want to observe that parameters in usage have a real effect on the network structure and that they actually
% behave as expected.
% For instance, by tuning $\mu_i$ over its extreme ranges, it enables to control the fraction of internal and external edges of single layer communities in $L_i$.
% Comparably, $p$ parameter is used to control coupling links within and outside multilayer communities.

\subsubsection{Experimental Setup}
% For the experiments, we fix the number of nodes $N_i$ in each layer $L_i$ at $100$ with maximum degree $k^{i}_{max}=10$ \& average
% degree $\langle k_i\rangle=6$. The power-law exponents for degree distribution ($\gamma_i$) and community size distribution ($\beta_i$)
% for each layer are fixed at 2 and 1 respectively and the coupling link density $d$ is fixed at $0.07$.
% which has proved to be reliable [22].
% Then we adopt the normalized mutual information (NMI), a relevant measure of similarity of partitions that
% allows us to quantify the difference between a base partition, say $\La$, and another partition $P$ delivered by one of the algorithms.
We implement the following state-of-the-art multilayer community detection algorithms to detect the communities in step (b).
% In order to achieve this evaluation, we first implement the following two representative algorithms for community detection in
% multilayer networks.

(i) MetaFac~\cite{metafac}: This algorithm detects communities based on the tensor factorization and requires the number of
  communities to be specified apriori\footnote{For our experiment, we vary the input number of communities from two to fifty and report
  the one exhibiting highest overlap
 with the ground truth.}. It detects communities in a way such that each of them contains at least one node from every 
 layer (i.e. only cross layer communities). 
%  Therefore,
%  MetaFac algorithm
%  intrinsically assumes that nodes of different types can be grouped into same number of communities.
%   For our experiment, we vary the input number of communities from two to fifty and report the one exhibiting highest overlap
%   with the ground truth.

(ii) CompMod~\cite{CompMod}: This algorithm detects communities by maximizing \emph{composite modularity}, which is a
  combination of the modularities of different subnetworks.

We compute normalized mutual information (NMI)~\cite{danon2005comparing} index to compare the detected communities with the ground truth
communities (step (b)). $NMI$ is a measure of similarity of communities, which attains a high value if the ground truth and detected
communities
exhibit good agreement.



\subsubsection{Evaluation}
Finally, we accomplish the step (c) by demonstrating that synthetic network possesses the desired properties, in the following way.

\textbf{Varying $\mu$:} In Fig.~\ref{eval_01} and Fig.~\ref{eval_02}, we increase the mixing
parameter $\mu$ (fraction of edges going out of single layer communities) which degrades the qualities of the communities.
Clearly, as $\mu$ increases, NMI drops for both the state of the art algorithms, irrespective of $\alpha$ and $p$ values.
This points to the fact that with increasing $\mu$, the ground truth communities degrade independent of the type of communities,
which meets our expectation.
% Regarding single layer communities, assessments are ommitted in the present paper but can be found in~\cite{lancichinetti2008benchmark}.
% For datasets consisting of both single layer and multilayer communties, wide range of values for $\mu$ and $p$ have been
% selected to explore
% different graph structures.\bigvee
% Accordingly, the community structure is regulated by $\alpha$,
% single layer and multilayer communities together are obtained with $\alpha = 0.4$ and multilayer communities only with $\alpha = 1.0$.
%Thus we can denote two types of synthetic datasets, $D_1$ and $D_2$ respectively.
%Every curve show the variation of NMI either with $\mu$ or $p$.
% \\Fig~\ref{eval_01} and Fig~\ref{eval_02} focus on the mixing parameter $\mu$, for both value of $\alpha$.
% As the reader can notice, when $\mu$ increases NMI drops subsequently for any $p$ value.

\textbf{Varying $p$:}
Similarly, Fig.~\ref{eval_03} and Fig.~\ref{eval_04} focuses on $p$, which intuitively regulates the cohesiveness of
the cross layer communities. Evidently, the obtained NMI rises with $p$ for both the community detection algorithms. The slope is
observed to be relatively steeper for the higher $\alpha$, as it indicates the presence of
more number of cross layer communities (magnifying the effect of $p$).
% how $p$ parameter affects the network structures composed of multilayer communities.
% In comparison $mu$ is still playing a key role, especially when $p$ is low
% , that is to say when multilayer communities are less defined by coupling links but more by intra-layer edges.
% Reasonably, $p$ parameter has little effect on decreasing the structure of single and multilayer communities (see~Fig\ref{eval_01}
% and Fig\ref{eval_03}).
% This is expected because most of communties are single layer ($60\%$) and thus coupling links are less significant.
% This is also expected and reinforce our confidence on this synthetic dataset generation.

%Hence, for both the parameters $\mu$ and $p$, the generated synthetic networks behave exactly the way they are expected to, confirming the robustness \& reliability of our proposed synthetic network generation mechanism. 